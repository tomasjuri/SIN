\documentclass[a4paper,11pt]{article}
\usepackage[margin=1in]{geometry}
\usepackage[czech,english]{babel}
\usepackage[utf8]{inputenc} %kodovani
\usepackage[T1]{fontenc}
\usepackage{cmap}
\usepackage{url}
\usepackage{graphicx}
\usepackage{hyperref} 
\usepackage{xcolor}
\hypersetup{
	colorlinks,
	linkcolor={red!50!black},
	citecolor={blue!50!black},
	urlcolor={blue!80!black}
}

\begin{document}

\begin{titlepage}
	\centering
	\includegraphics[width=0.15\textwidth]{fig/fit-zp2.pdf}\par\vspace{7cm}
	{\scshape\LARGE\bfseries Dopravní telematika\par}
	\vspace{0.5cm}
	{\scshape\Large dokumentace k projektu do předmětu SIN \par}
	\vspace{1.5cm}
	{\huge\bfseries \par}
	\vspace{2cm}
	{\Large\itshape \par}
	\vspace{8cm}
	Filip Denk, XDENKF00
	\\Tomáš Juřica, XJURIC22
\end{titlepage}


\section{TODO}
\begin{itemize}
	\item popsat SUMO
	\item popsat upravování mapy v JOSM (hlavně jaká je s tím jebačka)
	\item jaké je zadání a jaké jsme si dali cíle
	\item co popsat jednotlivá řešení tzn. 2 kruháče a jednu křižovatku
	\item popsat výstupy - statistika
	\item přepínání programů řízení - není potřeba, protože se to používá na denní časové úseky a my simulujeme jen hodinovou špičku
	\item konfiguraky do tomovych prikladu (pozor na nazvy bodu!), dokumentace, napsat do dokumentace, ze vsechny simulace jsou v hodinove spicce, popsat rozpocitane pomery mezi autem/truck/bus a popsat rozdil v cem se lisi (vizualne, accel, decel, ...)
	\item pridat obrazky (graf krizovatky z pdf)
\end{itemize}

\iffalse
\begin{figure}[!h]
	\centering
	\includegraphics[scale=0.34]{fig/graph.png}
	\caption{Graf komunikace.}
	\label{fig:marker}
\end{figure}

\subsection{Problémy}



\section{Návod ke spuštění}
\subsection{Potřebné prerekvizity}
\begin{itemize}
	\item ...

\end{itemize}


\section{Závěr}


\section{Struktura odevzdaného archivu}
\begin{itemize}
	\item \textbf{src/arduino} - zdrojový kód pro naprogramování Arduina
	\item \textbf{src/catkin/arm\_publisher} - package, ve kterém je node, který zpracovává data z
	MoveIt a přeposílá je na Arduino
	\item \textbf{src/catkin/rob\_arm} - konfigurační package pro MoveIt
	\item \textbf{arm\_description.urdf} - URDF model robotického ramene
	\item \textbf{dokumentace.pdf} - dokumentace k projektu
\end{itemize}

\fi

\end{document}
